\documentclass[a4paper, 12pt, titlepage]{article}

% osnovni paketi za jezik in kodiranje znakov
\usepackage[slovene]{babel} 
\usepackage[utf8]{inputenc}
\usepackage[T1]{fontenc}
\usepackage{lmodern}

% dodatni paketi
\usepackage{amsmath, amssymb, amsthm}

% priprava strani
\pagestyle{headings}
\author{Terezija Krečič}
\title{Nakup stanovanja}


\begin{document}

\maketitle
\tableofcontents
\newpage

%%%%%%%%%%%%%%%%%%%%%%%%%%%%%%%%%%%%%%%%%%%%%%
\section{Uvod}

Kaj ima nakup stanovanja skupnega z matematiko? Prva misel je lahko -- ``logično, opravka imamo z velikimi številkami, ki izginejo iz našega bančnega računa''. In seveda se ne motite. Vendar je tu še nekaj, s čimer se je po večini 2. polovico prejšnjega stoletja ukvarjalo že veliko znanih matematikov.

V nadaljevanju boste spoznali najbolj osnoven način reševanja svetovno znanega problema, znanega pod imenom \emph{``the secretary problem''}, ki se ukvarja z iskanjem najboljšega načina za izbor najboljšega elementa iz ponujene ponudbe. Izbira je omejena s pogojem, da si ogledujemo po en element naenkrat in se moramo na licu mesta odločiti, ali ga izberemo ali ne. To precej oteži reševanje, vendar nam prav zato vzbudi zanimanje in motivacijo, da bi ta izziv rešili, na koncu pa smo lahko nagrajeni z zadovoljstvom nad rešeno nalogo. Lahko se zgodi tudi, da nam to ni dovolj in se nam porodi še kakšno podobno vprašanje, dodaten ali pa milejši pogoj, in se potem sami lotimo nadaljnega reševanja.

Matematika dandanes veliko ljudem povzroča težave. Vse se začne že v šoli. Snov se kopiči, domače naloge so nerazumljive, učitelji nervozni \ldots Rado se zgodi, da so to glavni razlogi, da nekdo zasovraži matematiko in nič na svetu ga ne more prepričati, da je matematika lepa in lahko velikokrat olajša stvari. Primer: kmet ima 100 metrov ograje in bi z njo rad ogradil čim večji pašnik za koze. Če je kaj odnesel iz srednje šole, lahko na hitro zapiše tri vrstice računa z odvajanjem in dobi, da se mu najbolje splača očrtati kvadrat s stranico 25 metrov. Lahko mu pomagajo tudi izkušnje. Če pa nima ne enega ne drugega, kdo ve, za koliko trave bi prikrajšal svoje koze.

Ta seminar se prav tako ukvarja s praktičnim problemom iz vsakdanjega življenja. V ozadju se ne skriva skoraj nič matematične teorije, ki je ne bi spoznali na povprečni slovenski gimnaziji, zato je vsebina primerna tudi za vse nadebudne dijake, ki jim je matematika blizu. Če se ne počutite tako kompetentni, brez skrbi. Besedilo je zapisano tako, da je lahko razumljivo komurkoli, če se dovolj potrudi z branjem. Seveda pa je najbolj zanimiv rezultat in verjamem, da boste ta način reševanja kdaj tudi sami uporabili v praksi.

\newpage
%%%%%%%%%%%%%%%%%%%%%%%%%%%%%%%%%%%%%%%%%%%%%%
\section{Formulacija problema}

Zamislite si, da si kupujete novo hišo ali stanovanje. V časopisih in na internetu ste poiskali nek izbor stanovanj, vendar vas edina slika in kratek opis vsakega stanovanja seveda ne prepriča. Preden kupite novo stanovanje, si ga želite \emph{vsakega posebej} dobro ogledati. Vaša želja je, da bi si kupili \emph{najboljše} stanovanje iz izbora (naivno predpodstavimo, da denar ni ovira), zato se kmalu podate na ogled.

Najenostavnejši in tudi najefektivnejši način, s katerim bi poiskali najboljše stanovanje, je, da si ogledate vsa stanovanja iz ponudbe in se na koncu odločite za najboljšega. Žal se v realnem svetu človek na to težko zanese. Veste, da ogledi in ocenjevanje stanovanj vzamejo veliko časa, pa tudi samih ponudb stanovanj imate lepo število. Kdo ve, ali bo stanovanje, ki ste si ga ogledali mesec nazaj in vam je bilo precej všeč, še na voljo? Zato si postavite pravilo:

\begin{table}[h]
    \begin{tabular}{|p{13cm}|}
        \hline
Ogledujete si po eno stanovanje naenkrat. \textbf{Po vsakem ogledu se morate odločiti, ali boste to stanovanje kupili ali ne.} Pred odločitvijo si ne morete ogledati drugih stanovanj in ne veste, kakšna so. Če se odločite to stanovanje kupiti, se tu z ogledom ustavite, če pa se odločite, da si ogledate naslednje stanovanje, ponudbo trenutnega stanovanja izgubite. Torej \textbf{ni poti nazaj.}\\
        \hline
    \end{tabular}
    \caption{Pogoj za nakup stanovanja}
    \label{pogoj_za_nakup}
\end{table}


\subsection{Različice in drugi primeri}

Ta problem, ki se v tem seminarju imenuje \emph{nakup stanovanja}, je po svetu bolj znan kot \emph{``the secretary problem''}\footnote{problem tajnika, op. prev.}. Ima še veliko drugih imen: \emph{the marriage problem, the sultan's dowry problem, the fussy suitor problem, the googol game, the best choice problem}\footnote{poročni problem, problem sultanove dote, problem izbirčnega snubca, guglov problem (1 gugol je število, ki ji sledi sto ničel, torej $10^{100}$), problem najboljše izbire op. prev.} \ldots

Razložimo našteta imena na primerih:


Ostalih primerov je še veliko. V bistvu lahko pri vsakem izbiranju, kjer se upošteva podoben pogoj kot v \ref{pogoj_za_nakup}

\subsection{Matematična formulacija}


\subsection{Zgodovina}

%%%%%%%%%%%%%%%%%%%%%%%%%%%%%%%%%%%%%%%%%%%%%%
\section{Načini reševanja}
\subsection{Naključni izbor}
\subsection{Reševanje z vzorcem}
\subsubsection{Reševanje za male N}
\subsubsection{Reševanje za vzorcem velikosti K}
%%%%%%%%%%%%%%%%%%%%%%%%%%%%%%%%%%%%%%%%%%%%%%
\section{Rešitev za drugačne zahteve}

%%%%%%%%%%%%%%%%%%%%%%%%%%%%%%%%%%%%%%%%%%%%%%
\section{Zaključek}

Meni osebno so najbolj všeč naloge, ki matematiko povezujejo s praktičnim življenjem in reševanjem problemov in vsakdanjega življenja.
%%%%%%%%%%%%%%%%%%%%%%%%%%%%%%%%%%%%%%%%%%%%%%
\end{document}